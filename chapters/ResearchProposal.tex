\section{Research Proposal}
\subsection{Deterministic generation of a time-independent polarization-entangled photon pairs}
%	Restoring the Entanglement in the Biexciton-Exciton Radiative Cascade.
The main goal of my research is to provide a solution to the problem described in section \ref{biexciton-exciton-chapter}, where the existence of FSS in the two bright exciton eigenstates results in time-dependent phase oscillations and the reduction of entanglement in the biexciton-exciton radiative cascade.
\iffalse
The presence of the splitting in the lowest excitonic state is an unwanted effect in QDs that causes a degradation in the degree of entanglement between the two photons in the biexciton-exciton radiative cascade by lifting the degeneracy of the levels\cite{Winik2017}. This is due to several causes but mainly as a result of the asymmetry of the QD.$\newline$
\fi
Here, I propose a method to perform manipulation over the emitted photons instead of using general ideas and approaches to eliminate the FSS [many refs], which, so far, cannot do so. My method is unique in not trying to fine-tune the emitter's physical properties to restore the entanglement.\\
%While several methods were investigated to control the FFS, such as inducing strain and applying magnetic and electric fields, the problem persists.
%In this work, I will take a more straightforward approach by assuming that the FFS can not be eliminated completely and attempt to restore the entanglement after the photons emission as follows:
In this work, I will do the following steps to restore the entanglement:
\begin{enumerate}
	\item  I will improve and characterize the interferometer for this experiment, and I will search for better methods to perform the required manipulations over the emitted photons. 
	\item Using the interferometer, I will perform unitary rotation of the exciton's emitted photon polarization,  as theoretically described in Appendix \ref{appendix1}.
	\item I will extend the idea of manipulating the emitted exciton photon to both photons of the biexciton-exciton radiative cascade. Simultaneous manipulation of the two photons will restore the pair's polarization entanglement, as theoretically described in Appendix \ref{appendix2}.
\end{enumerate} 

\subsection{Additional proposed advances}
\subsubsection{Combining the radiative cascade into the knitting machine.}
\textcolor{red} {I'm not sure about the subject; requires more discussion. }
\subsubsection{Using the experimental system for feed-forward operations in 1D cluster states.}
I propose using our constructed device to perform feed-forward operations by rerouting each photon based on information gained about the polarization of the previously emitted photon. 
