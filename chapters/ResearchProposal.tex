\section{The Proposal}
\subsection{Problem Definition and Solution Approach.}
The presence of the splitting in the excitonic state causes a degradation in the degree of entanglement between the two photons in the biexciton-exciton radiative cascade \cite{Winik2017}.This       
\subsection{Theoretical formulation (exciton).}
\begin{equation}
	\ket{\Psi(t)} = \alpha(\ket{H_{XX} \otimes  H_X}\cdot e^{\frac{-iE_H t}{\hslash}}+\ket{V_{XX}\otimes V_X}\cdot e^{\frac{-iE_Vt}{\hslash}})
\end{equation}
\subsection{Theoretical formulation (biexciton-exciton)}
	\begin{equation}
	\ket{\Psi(t)} = \alpha(\ket{H_{XX} \otimes  H_X}\cdot e^{\frac{-iE_H t}{\hslash}}+\ket{V_{XX}\otimes V_X}\cdot e^{\frac{-iE_Vt}{\hslash}})
	\end{equation}
\begin{equation}
	\ket{\Psi(t)} = \alpha(\ket{H_{XX} \otimes H_X}+\ket{V_{XX}\otimes V_X}\cdot e^{\frac{-i(\triangle Et)}{\hslash}})
\end{equation}
\subsection{Rotation of the exciton's polarization.}
\subsection{Restoring the entanglement of the photons in the biexciton-exciton radiative cascade.}

If we construct two interferometers that allow us to induce phase shift to both the biexciton and exciton polarizations, then we can write the time-dependent phase relations as: 
\begin{equation}
\begin{aligned} 
		&\Phi_{H_{XX}}{(t,t_{prop})} = k_{H_{XX}}\cdot(t-t_{prop}) + \Phi^0_{H_{XX}} & \\	&\Phi_{V_{XX}}{(t,t_{prop})} = k_{V_{XX}}\cdot(t-t_{prop}) + \Phi^0_{V_{XX}} \\
        &\Phi_{H_{X}}{(t,t_{prop})} = k_{H_{X}}\cdot(t-t_{prop}) + \Phi^0_{H_{X}} &\\
		&\Phi_{V_{X}}{(t,t_{prop})} = k_{V_{X}}\cdot(t-t_{prop}) + \Phi^0_{V_{X}} 
\end{aligned}
\end{equation}
Here the  K's are the different slopes that introduce the shift to the photons' polarizations, and  $\Phi^0$'s are the initial phase of the photons at the time of emission. $t_{prop}$ is the propagation times of the photons from the quantum dot to the device. Since it's constant time we can simplify the function by including it in the constant phase $\Phi^0$.
By taking the starting time of our system as the biexciton excitation time, we can write the state using the $t_x$ and $t_{xx}$ ( where $t_{xx}$ and $t_{x}$ are the random emission times of the biexciton and exciton respectively), as follows:
\begin{equation}
\begin{split}
		\ket{\Psi(t)}= \alpha(\ket{H_{XX}}\cdot e^{i*\Phi_{H_{XX}}(t_{XX}-t^{xx}_{start})} \otimes 
		\ket{H_{X}}\cdot e^{i\Phi_{H_{X}}(t_{X}-t^{x}_{start})}+\\
        \ket{V_{XX}}\cdot e^{i\Phi_{V_{XX}}(t_{XX}-t^{xx}_{start})} \otimes 
		\ket{V_{X}}\cdot e^{i\Phi_{V_{X}}(t_{X}-t^{x}_{start})}\cdot e^{-i(\triangle Et)/\hslash})	
  \end{split}
\end{equation} 
next, plug it in
\begin{equation}
	\Psi(t) = \ket{H_{xx}H_x}+e^{i\Phi}\ket{V_{xx}V_x}
\end{equation}
where $\Phi(t)$ :
\begin{equation}
\begin{split}  
	\Phi(t) = (K_{V_{XX}}-K_{H_{XX}}+K_{V_X} - K_{H_X})\cdot t_{xx}+(K_{V_X}-K_{H_X} + \triangle E/\hbar)*t_x +\\
 (K_{V_{XX}} - K_{H_{XX}})\cdot-t^{XX}_{Start} +(K_{V_X}-K_{H_X})\cdot-t^X_{Start}+\\
 (\Phi^0_{V_{XX}}-\Phi^0_{H_{XX}}+\Phi^0_{V_{X}}-\Phi^0_{H_{X}}).
 \end{split}
\end{equation}
Conditions:
\begin{equation}
	K_{V_X}-K_{H_X} = -\triangle E/\hbar
\end{equation}
\begin{equation}
	(K_{V_{XX}} -K_{H_{XX}}) = -(K_{V_X}-K_{H_X})
\end{equation}
\begin{enumerate}
	\item 
	\item  Two
	\item  Three
\end{enumerate}
\subsection{Additional proposed advances}
\subsubsection{Combining the radiative cascade into the knitting machine.}
\subsubsection{Using the experimental system for feed-forward operations in 1D cluster states.}