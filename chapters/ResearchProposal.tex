\section{Research Proposal}
\subsection{Deterministic Generation of Time-Independent Polarization-Entangled Photon Pairs}
%	Restoring the Entanglement in the Biexciton-Exciton Radiative Cascade.
The main goal of my research is to provide a solution to the problem described in section \ref{biexciton-exciton-chapter}, where the existence of FSS in the two bright exciton eigenstates results in time-dependent phase oscillations and the reduction of entanglement in the biexciton-exciton radiative cascade.
\iffalse
The presence of the splitting in the lowest excitonic state is an unwanted effect in QDs that causes a degradation in the degree of entanglement between the two photons in the biexciton-exciton radiative cascade by lifting the degeneracy of the levels\cite{Winik2017}. This is due to several causes but mainly as a result of the asymmetry of the QD.$\newline$
\fi
Here, I propose a method to perform manipulation over the emitted photons instead of using general ideas and approaches to eliminate the FSS [many refs], which, so far, did not provide an adequate solution to the problem. My method, which uses an interferometer setup, is unique in not attempting to fine-tune the emitter's physical properties to restore the entanglement.\\
%While several methods were investigated to control the FFS, such as inducing strain and applying magnetic and electric fields, the problem persists.
%In this work, I will take a more straightforward approach by assuming that the FFS can not be eliminated completely and attempt to restore the entanglement after the photons emission as follows:
In this work, I will do the following steps to restore the entanglement:
\begin{enumerate}
	\item  I will improve and characterize the interferometer for this experiment, and I will search for better methods to perform the required manipulations over the emitted photons. 
	\item Using the interferometer, I will perform unitary rotation of the exciton's emitted photon polarization,  as theoretically described in Appendix \ref{appendix1}.

	\item I will extend the idea of manipulating the emitted exciton photon to both photons of the biexciton-exciton radiative cascade. Simultaneous manipulation of the two photons will restore the pair's polarization entanglement, as theoretically described in Appendix \ref{appendix2}.
\end{enumerate} 

\subsection{Additional Proposed Advances}
\subsubsection{Improving the Quality of a 1D Photonic Cluster States}
The generation of 1D photonic cluster state from QD has great potential, as demonstrated by \cite{Cogan2023}. However, this technique leads to small imperfections that induce some errors, such as the nonzero lifetime of the trion, which leads to the precession of the heavy-hole spin during emission \cite{Linder2009}. I propose using the experimental setup to reduce the errors and improve the quality cluster state. 
\subsubsection{Using the Experimental System for Feed-Forward Operations in the 1D  Photonic Cluster States.}
I propose using our constructed device to perform feed-forward operations to redirect each photon in a cluster state based on information gained about the polarization projection of the previously emitted photon. 
\subsubsection{Using the Optimized Platforms to Generate a Longer Cluster State.}
Generating a  Photonic 1D Cluster state from a QD is limited by a low collection efficiency and long emission lifetime \cite{Cogan2023}. I propose using a QD embedded in these new platforms described in section \ref{OptmizedPlatforms} to generate cluster states with 20 photons and higher. 
%\subsubsection{Combining the Radiative Cascade into the Knitting Machine.}
%\textcolor{red} {I'm not sure about the subject; requires more discussion. }