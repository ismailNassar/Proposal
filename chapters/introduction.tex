\section{Introduction}

%	\textcolor{red}{Change this section} \\
%	A qubit is the most elementary unit in quantum information. It is the quantum equivalent of the bit in classic computing and  can be described simply as a two-level quantum system. While a classic bit can take only one of two values described as 0 and 1, a qubit can exist in a superposition of these values, which allows it theoretically to have an infinite number of states.\\
%	There are multiple physical manifestations for qubits implemented in quantum systems today, such as charge in Josephson junction or the direction of the spin of an electron or the polarization state of a photon, and so on. Here in this work, we explore the usage of photonic qubits. The advantage of photonic qubits is the ability to manipulate them at room temperature and less prone to interact with their environment. To obtain photonic qubits, we need to have a single photon source which is one of the fundamental requirements in  quantum optical technologies.\\
%	One of the best candidates for such a source is self-assembled quantum dots (QDs). A QD can provide an optically controlled source of on-demand single photon source with well-defined polarization. We can use multiple methods in QD to define a two-level system; here, we choose the exciton and biexciton and study them by measuring the correlation between them.  
	
\subsection{Quantum Dots}
Self-assembled quantum dots (QDs) are nano-scale semiconductor structures that create a three-dimensional potential well, in both the valence and conduction band electrons and holes in three dimensions. Due to their small confinement length relative to the particle's wavelength, the energy levels of the QD are quantized with properties similar to atomic spectra which lead them to be described as "Artificial Atoms"\cite{Kastner1993}. \\
In recent years, QDs were thoroughly investigated as technology-compatible single photon sources\cite{Dekel2000}, providing a quantum source of "flying qubits" on demand. Moreover, recently it has been shown that QDs can emit pair of entangled photons and that an emitted photon can be entangled with the remaining spin in the QD.These achievements form the required building blocks for quantum information processing.\\
Populating the dot with electrons and holes is done with either resonant or non-resonant excitation. In resonant excitation an absorption of a photon excites an electron from the valence band to the conduction band where the photon energy is equal the difference in energy between these level. The missing electron in the valence band is treated as hole with a spin opposite in direction of the excited electron. This electron-hole pair form the bright exciton (BE) due to the Coulomb interaction between the negative electron and the positive hole. In non-resonant excitation, a strong laser excites the surrounding bulk material that generate electrons and holes in the vicinity of the QD.These charge carriers are free to move inside the semiconductor which can  be randomly trapped in the quantum dot potential. \\

The projection of the spin of the electron on the z-axis (growth axis) of the QD can be either 1/2 or -1/2 while for the heavy hole the spin projection is either 3/2 or -3/2 such that the two spin states of the bright exciton are $\ket{\frac{1}{2},-\frac{3}{2}}$ and $\ket{-\frac{1}{2},\frac{3}{2}}$  with a total spin in the z direction  of $\pm1$. While in case of electron-hole pair with parallel spin, the spin states are $\ket{\frac{1}{2},\frac{3}{2}}$ and $\ket{-\frac{1}{2},-\frac{3}{2}}$ with a total spin of $\pm2$.\\
In figure \ref{fig:energy_levels} we describe two of the simple configurations that we can have in a QD. We start with empty dot which we denote by $\ket{0}$.
\begin{figure}[H]
	\centering
	\includegraphics[scale=1]{figures/energy-levels.png}
	\caption{Schematic of the energy levels in the quantum dot for empty quantum dot (a),exciton (b) and a biexciton (c)}
	\label{fig:energy_levels}
\end{figure}

while having two electron-hole pairs falling in the dot a biexciton is formed and we denote it by $\ket{{XX}_0}$. The total energy of the biexciton differs from twice the energy of the exciton due to the interaction between all the particles.\\

In an ideal QD, the radiative decay path of the biexciton back to the ground state goes via one path, but due to the anisotropy of the QD, the energy of exciton is split in what is defined as the fine structure splitting (FSS) as seen in figure \ref{fig:Decay_paths}b. Here we refer to the up (down) spin of the electron as $\ket{\uparrow}$ ($\ket{\downarrow}$ and the up (down) spin of the hole as $\ket{\Uparrow}$ ($\ket{\Downarrow}$ such as the two eigenstates of the exciton are  $\ket{\uparrow\Downarrow}$ and $\ket{\downarrow\Uparrow}$,and the decay from these states result in the emission of photons with co-linear polarization. Here we refer to them as horizontal $\ket{H}$ and vertical $\ket{V}$. We can represent these two rectilinear bases using the Bloch sphere where the poles in the sphere are $\ket{H}$ and $\ket{V}$ base.\\

In addition to the rectilinear polarization states,We can define the diagonal linear and
the circular polarization bases: 
\begin{equation}
	\begin{split}
		\ket{L}=(\ket{H}+i\ket{V})/\sqrt{2} \\
		\ket{R}=(\ket{H}-i\ket{V})/\sqrt{2}
	\end{split}
\end{equation}
where $\ket{R}$ and $\ket{L}$ are the left and right circular bases respectively, and:
\begin{equation}
	\begin{split}
		\ket{D}=(\ket{H}+\ket{V})/\sqrt{2}\\
		\ket{\bar{D}}=(\ket{H}+\ket{V})/\sqrt{2}
	\end{split}
\end{equation}
$\ket{D}$ and $\ket{\bar{D}}$ are the diagonal and anti-diagonal bases respectively

\begin{figure}[H]
	a)
	\raggedleft
	\def\psiLat{0}
	\def\psiLon{-50}
	\begin{blochsphere}[radius=2.5 cm,tilt=20,rotation=-20,opacity=0]
		\labelLatLon{psi}{\psiLat}{-\psiLon};
		\draw[-latex] (0,0) -- (psi) node[above]{$\large\ket{\psi}$};
		\drawRotationLeft[scale=0.9,style={red}]{0}{0}{0}{15}
		
		\drawGreatCircle[style={dashed}]{0}{0}{0}
		\drawBallGrid[style={opacity=0.5}]{30}{30}
		\draw [fill] (0,0) circle (1.5pt);
		\drawGreatCircle[style={dashed}]{0}{0}{0}
		\labelLatLon{up}{90}{0};
		\labelLatLon{down}{-90}{90};
		\node[above] at (up) {{ $\ket{H}$ }};
		\node[below] at (down) {{ $\ket{V}$}};
		\node at (2.8,0) {{$\ket{R}$}};
		\node at (-2.8,0) {{$\ket{L}$}};
		\node at (0,1) {{$\ket{D}$}};
		\node at (0,-1) {{$\ket{\bar{D}}$}};
	\end{blochsphere}
	b)
	\raggedright
	\includegraphics[scale=0.8]{figures/Decay_paths.png}
	\caption{a) A Bloch sphere representation of the spin state. A point on the sphere represents an arbitrarily polarized spin state. b) Decay paths of the biexciton back the ground state where the red arrows represent the emission of a H polarized photon and blue arrows represent the emission of V polarized photon.}
	\label{fig:Decay_paths}
\end{figure}
When the biexciton spontaneously radiatively decays it  emits a photon leaving 
in the QD an exciton in coherent superposition of its two eigenstates. The optical
selection rules for the biexciton radiative recombination and the lack of information by “which path” the recombination proceeds result in entanglement between the exciton state and the polarization state of the emitted photon. Their mutual wave function is given by:
\begin{equation}
	\ket{\psi_{X^0}} = \frac{1}{\sqrt{2}}(\ket{H_1H^1_H} +\ket{V_1V^1_V})
\end{equation}
here $\ket{H_1}$ and $\ket{V1}$ are the two rectilinear polarization states of the first (biexciton) photon. since the two exciton eigenstates are not degenerate, the relative phase between these eigenstates precesses in time with a period of $T_P = h/\triangle$, where where $h$ is the Planck constant and $\triangle$ is the exciton FSS \cite{rwinik2017}.
This precession is schematically described on the exciton Bloch sphere in figure. $\ref{fig:Decay_paths}a$.
The precession “stops” when the exciton recombines and the radiative cascade is completed with the emission of a second photon. The two photons are thus entangled. Their mutual wave function depends on the recombination time and is given by:
\begin{equation}
	\ket{\psi_{X^0}} = \frac{1}{\sqrt{2}}(\ket{H_1H_1} +e^{-i2\pi t/T_P}\ket{V_1V_2})
\end{equation}
where $\ket{H_2}$ and $\ket{V_2}$ are the second (exciton) photon polarization states and $t = t_{X_0} - t_{{XX}_0}$ is the time between the emission of the biexciton photon $t_{{XX}_0}$ and that of the exciton $t_{X_0}$.


