\section{Introduction}
%	\textcolor{red}{Change this section} \\
%	A qubit is the most elementary unit in quantum information. It is the quantum equivalent of the bit in classic computing and  can be described simply as a two-level quantum system. While a classic bit can take only one of two values described as 0 and 1, a qubit can exist in a superposition of these values, which allows it theoretically to have an infinite number of states.\\
%	There are multiple physical manifestations for qubits implemented in quantum systems today, such as charge in Josephson junction or the direction of the spin of an electron or the polarization state of a photon, and so on. Here in this work, we explore the usage of photonic qubits. The advantage of photonic qubits is the ability to manipulate them at room temperature and less prone to interact with their environment. To obtain photonic qubits, we need to have a single photon source which is one of the fundamental requirements in  quantum optical technologies.\\
%	One of the best candidates for such a source is self-assembled quantum dots (QDs). A QD can provide an optically controlled source of on-demand single photon source with well-defined polarization. We can use multiple methods in QD to define a two-level system; here, we choose the exciton and biexciton and study them by measuring the correlation between them.  
		Self-assembled quantum dots (QDs) are nano-scale semiconductor structures that can confine electrons and holes in three dimensions. Due to their small confinement length relative to the particle's wavelength, the energy levels of the QD are quantized with properties similar to atoms which lead them to be described as "Artificial Atoms"\cite{Kastner1993}. \\



