\section*{Abstract}
We present a research proposal for the doctoral thesis to investigate the emission of entangled photon pairs from the biexciton-exciton radiative cascade in a self-assembled quantum dot. We propose a new experimental technique in which we manipulate the polarization state of the emitted photon to increase the degree of entanglement of the photon pair.
This technique, however, is not restricted to the biexciton-exciton radiative cascade and can be used in various experimental applications for fast manipulation of photon polarization in general and in particular for the generation of entangled photons. I present a theoretical description and propose an experimental setup to manipulate the photon's polarization.
Demonstrating a higher degree of entanglement using our technique will enable further studies using entangled photon pairs from quantum emitters, as well as using the photon pair in quantum communication protocols.

%We propose a new experimental technique, that was never implemented before, to enhance the degree of entanglement of the photon pair.
%Furthermore, the proposed technique is applicable to various experimental approaches for the generation of entangled photons.
%In addition, I present a theoretical description of the technique and present preliminary results for achieving my goals.\\
%This investigation will enhance our capability to manipulate the emission of entangled photons from quantum dots and aid in their application as a provider of a qubit system for quantum information processing.
%\textcolor{red}{We present a research proposal for the doctoral thesis to investigate the emission of entangled photon pairs from the biexciton-exciton radiative cascade from a self-assembled quantum dot. This proposal will investigate the entanglement between the two photons and provide a new experimental technique that was never implemented before to enhance their degree of entanglement. Furthermore, It's applicable to multiple approaches for generating entanglement photons from the QD. We also present a theoretical description of the technique and the initial development of this technique alongside preliminary results.\\
 %This investigation will enhance our capability to manipulate the emission of entangled photons from quantum dots and aid in their application as a provider of a qubit system for quantum information processing.}