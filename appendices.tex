\section{Appendix I}\label{appendix1}
\subsection*{Theoretical formulation (exciton).}
We start by writing the exciton's wavefunction after a short laser $\pi$-pulse excitation with polarization $\ket{P}=\alpha\ket{H_L}+\beta\ket{V_L}$, where $\alpha^2+\beta^2=1$:
\begin{equation}
	\ket{\Psi_X(t)} = \alpha\ket{X_H}\cdot e^{\frac{-iE_H t}{\hslash}}+\beta\ket{X_V}\cdot e^{\frac{-iE_Vt}{\hslash}}
\end{equation}
Here $\ket{X_H}$ and $\ket{X_V}$ are the bright exciton eigenstates, and $E_{H,V}$ are the related eigenenergies.
By defining $\hbar\omega=\Delta E=E_V-E_H$, dropping global phase, and rearranging we get:
\begin{equation}
	\ket{\Psi_X(t)} = \alpha\ket{X_H} + \beta\ket{X_V}\cdot e^{-i\omega t}
\end{equation}
The instantaneous state of the emitted photon's polarization is directly related to the excitonic wavefunction \textcolor{red}{[ref]}:
\begin{equation}
	\ket{\Psi_P(t)} = \alpha\ket{H_X}+\beta\ket{V_X}\cdot e^{-i\omega t}
\end{equation}
Here, $\ket{H_X}$ and $\ket{V_X}$ are the photon's polarization state.
We note that in \textcolor{red}{eqs. 7 and 8} we split the exciton and photon wavefunction and did not took into account the radiative decay.
We further note that the instantaneous polarization state of the emitted photon change in time with the same angular frequency as the exciton.

%The full wavefunction of the exciton and photon reads:
%\begin{equation}
%	\ket{\Psi(t)} = e^{-t/2\tau}\ket{\Psi_X(t)} + \left(1-e^{-t/\tau}\right)^\frac{1}{2}\ket{\Psi_P(t)}
%\end{equation}

%The state of a photon emitted from exciton excitation is described as:
%\begin{equation}
%	\ket{\Psi(t)} = \alpha\ket{H_X}\cdot e^{\frac{-iE_H t}{\hslash}}+\beta\ket{V_X}\cdot e^{\frac{-iE_Vt}{\hslash}}
%\end{equation}
%Here $E_{H,V}$ are the energy levels, $H_X$ and $V_X$ are the horizontal and vertical respectively, and $\alpha$,$\beta$ are coefficients such that $\alpha^2 +\beta^2 = 1$.This 
%can be rewritten as:
%\begin{equation}
%	\ket{\Psi(t)} = \alpha\ket{H_X}+\beta\ket{V_X}\cdot e^{\frac{-i\triangle E t}{\hslash}}
%\end{equation}
%Where $\triangle E = E_H -E_V$. This can be simplified by replacing $ \frac{\triangle E}{\hslash}$ with $\omega$:
%\begin{equation}
%	\ket{\Psi(t)} = \alpha\ket{H_X}+\beta\ket{V_X}\cdot e^{-i\omega t}
%\end{equation}
\subsection*{Rotation of the exciton's polarization.}


If we construct a device that allows us to induce a time-dependent phase shift to the exciton polarization, then we can write phase relations as:
\begin{equation} \label{exciton_phases}
	\begin{aligned} 
		&\Phi_{H_{X}}{(t,t_{prop})} = K_{H_{X}}\cdot(t-t_{prop}) + \Phi^0_{H_{X}} &\\
		&\Phi_{V_{X}}{(t,t_{prop})} = K_{V_{X}}\cdot(t-t_{prop}) + \Phi^0_{V_{X}} 
	\end{aligned}
\end{equation}
Here the  K's are the different slopes that introduce a shift to the photons' polarizations, and  $\Phi^0$'s are the initial phase of the photons. $t_{prop}$ is the propagation times of the photons from the quantum dot to the device. Since it's constant time, we can simplify the function by including it in the constant phase $\Phi^0$.
\begin{equation}
	\begin{split}
		\ket{\Psi(t)}= \alpha( \ket{H_{X}}\cdot e^{i\Phi_{H_{X}}(t_{X}-t^{x}_{start})})+
		\beta (\ket{V_{X}}\cdot e^{i\Phi_{V_{X}}(t_{X}-t^{x}_{start})}\cdot e^{-i\omega t})
	\end{split}
\end{equation} 
Where $t_x$ is the random emission time of the exciton photon, and $t_{start}$ is the time when ramping of the differential phase begins. Substituting relations (\ref{exciton_phases}) and reorganizing the terms, we get:
\begin{equation}
	\ket{\Psi(t)} = \alpha\ket{H_X}+\beta\ket{V_X}\cdot e^{i\Phi}
\end{equation}
where $\Phi(t)$ :
\begin{equation}
	\begin{split}  
		\begin{aligned} 
			\Phi(t) = &(K_{V_X}-K_{H_X} + \omega)\cdot t_x -\\
			&t^X_{Start} \cdot(K_{V_X}-K_{H_X})+\\
			&(\Phi^0_{V_{X}}-\Phi^0_{H_{X}})
		\end{aligned}
	\end{split}
\end{equation}
For the function to be independent of the time $t_x$, the following condition must be met:
\begin{equation}
	K_{V_X}-K_{H_X} =-\omega 
\end{equation}
Since we are interested only in the relative phase, we can further reduce the condition by introducing phase shift to only one direction:
\begin{equation}
	K_{V_X} =-\omega 
\end{equation}
\section{Appendix II} \label{appendix2}
\subsection*{Theoretical formulation (biexciton-exciton)}
The state of the two photons emitted from the radiative cascade is described as follows:
\begin{equation}
	\ket{\Psi(t)} = \alpha(\ket{H_{XX} \otimes  H_X}\cdot e^{\frac{-iE_H t}{\hslash}}+\ket{V_{XX}\otimes V_X}\cdot e^{\frac{-iE_Vt}{\hslash}})
\end{equation}
Where $HH_X$ and $HH_V$ are the horizontal and vertical polarization of the biexciton, $\alpha$ is a global phase. Same with the exciton, we can simplify the function to:
\begin{equation}
	\ket{\Psi(t)} = \alpha(\ket{H_{XX} \otimes H_X}+\ket{V_{XX}\otimes V_X}\cdot e^{-i\omega t})
\end{equation}
\subsection*{Restoring the entanglement of the photons in the biexciton-exciton radiative cascade.}
Similar to the treatment of the exciton treatment, We can construct two devices that allow us to induce phase shift to both the biexciton and exciton polarizations, The time-dependent phase relations to both photons are: 
\begin{equation} \label{biexciton_phases}
	\begin{aligned} 
		&\Phi_{H_{XX}}{(t,t_{prop})} = K_{H_{XX}}\cdot(t-t_{prop}) + \Phi^0_{H_{XX}} & \\	&\Phi_{V_{XX}}{(t,t_{prop})} = K_{V_{XX}}\cdot(t-t_{prop}) + \Phi^0_{V_{XX}} \\
		&\Phi_{H_{X}}{(t,t_{prop})} = K_{H_{X}}\cdot(t-t_{prop}) + \Phi^0_{H_{X}} &\\
		&\Phi_{V_{X}}{(t,t_{prop})} = K_{V_{X}}\cdot(t-t_{prop}) + \Phi^0_{V_{X}} 
	\end{aligned}
\end{equation}
Here the K's are the different slopes that introduce the shift to the photons' polarizations, and  $\Phi^0$'s are the initial phase of the photons. $t_{prop}$ is the propagation times of the photons from the quantum dot to the device. Since it's constant time, we can simplify the function by including it in the constant phase $\Phi^0$.
By taking the starting time of our system as the biexciton excitation time, we can write the state using the $t_x$ and $t_{xx}$ ( where $t_{xx}$ and $t_{x}$ are the random emission times of the biexciton and exciton respectively), as follows:
\begin{equation}
	\begin{split}
		\ket{\Psi(t)}= \alpha(\ket{H_{XX}}\cdot e^{i\Phi_{H_{XX}}(t_{XX}-t^{xx}_{start})} \otimes 
		\ket{H_{X}}\cdot e^{i\Phi_{H_{X}}(t_{XX}+t_{X}-t^{x}_{start})}+\\
		\ket{V_{XX}}\cdot e^{i\Phi_{V_{XX}}(t_{XX}-t^{xx}_{start})} \otimes 
		\ket{V_{X}}\cdot e^{i\Phi_{V_{X}}(t_{XX}+t_{X}-t^{x}_{start})}\cdot e^{-i\omega t})	
	\end{split}
\end{equation} 
Substituting relations (\ref{biexciton_phases}) and reorganizing the terms, we get:
\begin{equation}
	\Psi(t) = \ket{H_{xx}H_x}+e^{i\Phi}\ket{V_{xx}V_x}
\end{equation}
where $\Phi(t)$ :
\begin{equation}
	\begin{split}  
		\begin{aligned} 
			\Phi(t) = &(K_{V_{XX}}-K_{H_{XX}}+K_{V_X} - K_{H_X})\cdot t_{xx}+(K_{V_X}-K_{H_X} + \omega)\cdot t_x -\\
			&t^{XX}_{Start} \cdot(K_{V_{XX}} - K_{H_{XX}})-t^X_{Start} \cdot(K_{V_X}-K_{H_X})+\\
			&(\Phi^0_{V_{XX}}-\Phi^0_{H_{XX}}+\Phi^0_{V_{X}}-\Phi^0_{H_{X}})
		\end{aligned}
	\end{split}
\end{equation}
For the function to be independent of the times $t_{xx}$ and $t_{x}$, the following condition must be met:
\begin{equation}
	K_{V_X}-K_{H_X} = -\omega
\end{equation}
\begin{equation}
	(K_{V_{XX}} -K_{H_{XX}}) = -(K_{V_X}-K_{H_X})
\end{equation}
The same with the exciton rotation treatment, We can further simplify the conditions by inducing a shift in one direction:
\begin{equation}
	K_{V_X} = -\omega
\end{equation}
\begin{equation}
	K_{V_{XX}} = -K_{V_X}
\end{equation}